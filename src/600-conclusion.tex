% !TEX root = document.tex

\chapter{\label{chap:conclusion}Conclusion}

In this thesis we have examined how CPS can be improved upon as an IR in three ways: simpler control flow specification, typed transformations, and a modular interface. We have proposed command trees as a solution, which help both language implementers and compiler writers. Our new compilation scheme is validated by using it to implement LamToWat, a compiler that translates the lambda calculus into WebAssembly.

We started by discussing a refence implementation of LamToWat and showed how the source code is transformed by CPS conversion, closure conversion, and emitting. We identified three shortcomings of the IR: complex specification of CPS conversion, no types to show that functions are not nested, and constructor duplication when emitting.

We proposed command trees as the solution to these shortcomings. There are two types of command trees: semantic and syntactic. We had to split our data type because the semantic command trees did not allow for variable analysis, which is essential to closure conversion. We conjecture the possibility of an isomorphism between the two types of command trees.

The monadic nature of command trees allowed us to bind commands together and write the CPS conversion step easily. Open unions provided a modular approach to compilation by allowing the programmer to extend her set of commands. By having each transformation change specific command modules we clearly indicated what it altered. The implementation of command trees in this thesis can be extended in many aspects. We proposed some improvements of command trees, such as type constraints to show the absence of free variables or open unions that behave more like real sets.

We tested both implementations of LamToWat by compiling and running programs in the lambda calculus and checking their results. We listed a number of improvements with respect to the compiler itself: optimized closures and the inlining of functions.