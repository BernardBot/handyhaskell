% !TEX root = document.tex

\chapter{\label{chap:cpscomp}Compiling with Continuations}
In this chapter we will develop the LamToWat compiler that translates the lambda calculus into WebAssembly \autocite{webassemblyhomepage}. Our compiler will be written in Haskell \autocite{haskellhomepage}. We follow a minimal version of the approach by \citeauthor{DBLP:books/daglib/0022396}. Consequently, \ac{CPS} will be the \ac{IR}. The purpose of building LamToWat is to examine and then improve its \ac{IR}. Although we are not concerned with the peripherals of the compiler, an \ac{IR} is not used in a vacuum. We want to examine the compiler steps that map to and from the \ac{IR} as well as the ones that map to the \ac{IR} itself. If we can improve upon the already favorable features of \ac{CPS}, we are creating a better IR to program in.

What makes \ac{CPS} favorable as an \ac{IR} is that it makes control flow and data flow explicit. These features nicely represent the objective of a compiler. The process of translating from a high-level to a low-level language can be viewed as describing abstractions in finer detail. \ac{CPS} uses special functions, called 'continuations', to describe more complex control flow constructs. Continuations are special functions. This fact becomes essential when translating languages with first-class functions.

LamToWat is a multipass compiler. This means that multiple passes over the \ac{IR} will be made, each transforming a part of the representation. The complete compiler comprises these \ac{IR} transformations with additional passes to read in source files and print the resulting \ac{WA}.

\begin{equation*}
  String \xto{lexify} Token \xto{parsify} Exp \xto{cpsify} Cexp \xto{closify} Cexp \xto{hoist} Cexp \xto{emit} Wmodule \xto{show} String
\end{equation*}

LamToWat is split into a front-end and a back-end. The front-end of the compiler is made up of lexing, parsing, and \ac{CPS} conversion. We will not discuss lexing and parsing in detail, as it is irrelevant to the research in this thesis. We use \icode{Alex}\autocite{haskellalex} and \icode{Happy}\autocite{haskellhappy} to generate both a lexer and a parser for our compiler, respectively. The lexer converts source files into tokens and the parser in turn converts these tokens into an \ac{AST}. The back-end of LamToWat consist of three transformations on the \ac{IR}: closure conversion, hoisting, and emitting.

The following sections will first describe the data types used in LamToWat with extra attention to the \ac{CPS} \ac{IR}. Then we will adhere to the organization of the compiler and discuss the relevant compiler passes over these data types in order.

\section{\label{section:datatypes}Data Types}
The data types of LamToWat will be implemented as Haskell data types. Haskell data types are indicated with the \icode{data} keyword. Every datatype has a name, which comes after the keyword. Then a number of constructors follow separated by vertical bars. Data types can be recursively defined. Type aliases indicated with the \icode{type} keyword are also used; providing new names for already declared data types.

\subsection{\label{subsection:expdata}Lambda Calculus}
\begin{lstlisting}[language=Haskell]
type Var = String

data Exp
  = Abs Var Exp
  | App Exp Exp
  | Var Var
  | Num Int
  | Add Exp Exp
\end{lstlisting}

The \icode{Exp} datatype represents the \ac{AST} of the lambda calculus. The three constructors for \icode{Abs}traction, \icode{App}lication, and \icode{Var}iables encompass the standard definition of the lambda calculus. Two constructors for \icode{Num}bers and \icode{Add}ition are added to make basic arithmetic possible. Although the language is very small it does support first-class functions and recursion. Moreover, the lambda calculus is well-studied \autocite{barendregt1984lambda} and many pathological programs exist.

\subsection{\label{subsection:cpsdata}CPS Language}
\begin{lstlisting}[language=Haskell]
type Var = String

data Val
  = VAR Var
  | NUM Int
  | LABEL Var

type Fun = (Var, [Var], Cexp)
data Cexp
  = FIX [Fun] Cexp
  | APP Val [Val]
  | ADD Val Val Var Cexp
  | RECORD [Val] Var Cexp
  | SELECT Int Val Var Cexp
  | DONE Val
\end{lstlisting}

In this thesis we follow the \ac{CPS} language datatype definition by \citeauthor{DBLP:books/daglib/0022396} with some minor adjustments. It is used as \ac{IR} because it facilitates compiler transformations relevant to functional programming languages. The datatype is similar to the control flow graph of a program. Control flow is modeled with functions and function calls. Data flow is modeled with records. LamToWat uses \ac{CPS} to transform nested, higher-order functions to simple functions.

The \icode{CPS} datatype guarantees that compiler transformations can be easily implemented by enforcing the following properties.

\begin{itemize}
\item Functions have an extra argument for their continuation.
\item Functions do not return, instead the last thing a function does is call a continuation.
\item Function parameters can only be values.
\item All intermediate values have names.
\end{itemize}

\icode{CPS} functions are made up of a name, argument names, and a body. They are defined in mutually recursive \icode{FIX}ed function blocks. Values are either \icode{NUM}bers, \icode{VAR}iables, or \icode{LABEL}s. \icode{LABEL}s are used for proper function names appearring in \icode{FIX}es, \icode{VAR}iables for everything else including continuations. The \icode{CPS} language enforces the the distinction between \icode{Val}ues and \icode{Cexp}ressions by putting them in separate data types. All expressions except \icode{APP} and \icode{DONE} have a continuation as their last argument. Since the \icode{ADD}, \icode{RECORD}, and \icode{SELECT} expressions produce a value, they name it before continuing. \icode{DONE} helps us to define the function that translates a source language into \ac{CPS}. As this expression represents a return statement, it violates the second property of \ac{CPS}. No problems are cause by it, because we will make sure that it is the final expression of our programs. This makes it a global return of the program itself, not of a function.

Variable names in \ac{CPS} are syntactically represented by simple strings. This is a design choice as it allows for nonsense programs that use undefined variables, but also for easy manipulation of variables and generation of variable names.

\subsection{\label{subsection:webdata}WebAssembly Language}
\begin{lstlisting}[language=Haskell]
type Size = Int
type Offset = Int
type Var = String

data GlobalType = Mut Wtype | Immut Wtype
data Wtype = I32

data Wmodule = Wmodule [Wtexp]

data Wtexp
  = Func Var [(Var, Wtype)] Wtype [(Var, Wtype)] [Wexp]
  | TypeDef Var [Wtype] Wtype
  | GlobalDef Var GlobalType Wexp
  | Memory Size
  | Table Size
  | Elem Wexp [Var]
  | Export Var Var

data Wexp
  = I32_Const Int
  | Return Wexp
  | LocalGet Var
  | LocalSet Var Wexp
  | GlobalGet Var
  | GlobalSet Var Wexp
  | Type Var
  | Call_Indirect Wexp Wexp [Wexp]
  | Return_Call_Indirect Wexp Wexp [Wexp]
  | Primop Wop [Wexp]
  | Load Offset Wexp
  | Store Offset Wexp Wexp

data Wop = PLUS
\end{lstlisting}

The WebAssembly language is a typed, hierarchical, high-level assembly language and a subset of WebAssembly. As the final language in the compilation process it gives some guarantees about the well-formedness of the output of the compiler. The fundamental unit of code is a \icode{Wmodule} made up of a list of \icode{Wtexp}s. A \icode{Wtexp} is made up of \icode{Wexp}s.

\icode{I32_Const}, 32-bit integers are the basic values, used as both numbers and pointers. \icode{Func}tions are defined at the module level and can not be nested. They take zero or more arguments to a result as declared in their type signature. Local variables used in a function's body are also declared in its signature and can be \icode{LocalSet} and \icode{LocalGet}. All function calls are indirect, meaning that the function a variable points to is resolved at runtime. A \icode{Table} with type signature \icode{Elem}ents for the called functions is required at compile time. Persistent \icode{Memory} is allocated statically and its data can be \icode{Load}ed and \icode{Store}d. A \icode{GlobalDef}ined variable can be used in the entire module, it has a \icode{GlobalType} and can be \icode{GlobalSet} and \icode{GlobalGet}. The only \icode{Primop}eration is addition.

\section{\label{section:transforms}Transformations}
\subsection{\label{subsection:cpsconvert}CPS Conversion}
\begin{lstlisting}[language=Haskell]
convert :: Exp -> (Val -> Gen Cexp) -> Gen Cexp
convert (Var x) c = c (VAR x)
convert (Num i) c = c (NUM i)
convert (Abs x e) c = do
  f <- fresh "f"
  k <- fresh "k"
  c' <- c (LABEL f)
  cf <- convert e (\ z ->
         return $ APP (VAR k) [z])
  return $ FIX [(f,[x,k],cf)] c'
convert (App f e) c = do
  r <- fresh "r"
  x <- fresh "x"
  c' <- c (VAR x)
  cf <- convert f (\ f' ->
         convert e (\ e' ->
           return $ APP f' [e', LABEL r]))
  return $ FIX [(r,[x],c')] cf
convert (Add a b) c = do
  x <- fresh "x"
  c' <- c (VAR x)
  convert a (\ a' ->
   convert b (\ b' ->
    return $ ADD a' b' x c'))
\end{lstlisting}

\subsection{\label{section:closconvert}Closure Conversion}
\begin{lstlisting}[language=Haskell]
closify e = go e
  where go (RECORD vs x e)  = RECORD vs x (go e)
        go (SELECT i v x e) = SELECT i v x (go e)
        go (ADD v1 v2 x e)  = ADD v1 v2 x (go e)

        go (DONE v)         = close [v] $ DONE (renameC v)
        go (APP v vs)       = close (v:vs) $ (apply v) (map renameC vs)
        go (FIX fs e)       = FIX (map goFunc fs) (go e)

        goFunc (f,as,b) = (f,envName:as,open (lookdvs f) (go b))

        dvs = dv e []
        lookdvs f = fromJust $ lookup f dvs

        renameC (LABEL f) = VAR (f ++ closTag)
        renameC v = v

        closTag = "_C"
        funcTag = "_F"
        envName = ".env"

        close [] k = k
        close (LABEL f:vs) k = 
          RECORD (LABEL f : map VAR (lookdvs f)) (f ++ closTag) (close vs k)
        close (_:vs) k = close vs k

        open xs k = foldr (\ (i,x) -> SELECT i (VAR envName) x) k (zip [1..] xs)

        apply (LABEL f) vs = APP (LABEL f) (VAR (f ++ closTag):vs)
        apply (VAR f) vs = SELECT 0 (VAR f) f_F (APP (VAR f_F) (VAR f:vs))
          where f_F = f ++ funcTag
\end{lstlisting}
\subsection{\label{subsection:hoist}Hoisting}
\begin{lstlisting}[language=Haskell]
hoist :: Cexp -> Cexp
hoist (RECORD rs x e)  = case hoist e of FIX fs e -> FIX fs (RECORD rs x e)
hoist (SELECT i v x e) = case hoist e of FIX fs e -> FIX fs (SELECT i v x e)
hoist (ADD v1 v2 x e)  = case hoist e of FIX fs e -> FIX fs (ADD v1 v2 x e)
hoist (APP v vs)       = FIX [] (APP v vs)
hoist (DONE v)         = FIX [] (DONE v)               
hoist (FIX fs e)       = case hoist e of FIX fs e -> FIX (fs'++fs) e
  where fs' = concatMap (\ (f,as,b) -> case hoist b of FIX fs e -> (f,as,e) : fs) fs
\end{lstlisting}
\subsection{\label{section:emit}Emitting}
\begin{lstlisting}[language=Haskell]
emit :: Cexp -> Wmodule
emit (FIX fs e) = Wmodule $
  heapAlloc : 
  heapPointerDef :
  tableAlloc :
  tableElem :
  typeDefs ++
  startFunc :
  startExport :
  funcDefs
  where prefix = "."

        heapSize = 1
        heapPointer = prefix ++ "heap_pointer"
        heapInitLoc = 0
        heapAlloc = Memory heapSize
        heapPointerDef = GlobalDef heapPointer (Mut I32) (I32_Const heapInitLoc)

        tableSize = length fs
        tableAlloc = Table tableSize
        tableElem = Elem (I32_Const 0) fns
        fns = map (\ (f,_,_) -> f) fs

        typePrefix = prefix ++ "t"
        typeDefs = map emitTypeDef fls
        emitTypeDef n = TypeDef (typePrefix ++ show n) (replicate n I32) I32

        fls = nub $ map (\ (_,as,_) -> length as) fs 
        
        startFuncName = prefix ++ "start"
        startFunc = Func startFuncName [] I32 (mapi32 (lc e)) (emitE e)
        startExport = Export startFuncName startFuncName
        mapi32 = map (\ x -> (x,I32))
        
        funcDefs = map emitF fs
        emitF (f,as,b) = Func f (mapi32 as) I32 (mapi32 (looklvs f)) (emitE b)

        looklvs f = fromJust $ lookup f lvs

        lvs = map (\ (f,as,b) -> (f,nub (lc b))) fs
        
        intSize = 4

        emitE (RECORD vs x e) =
          zipWith (\ v i -> Store i (GlobalGet heapPointer) (emitV v)) vs [0,intSize..] ++
          LocalSet x (GlobalGet heapPointer) :
          GlobalSet heapPointer (Primop PLUS [GlobalGet heapPointer, I32_Const (length vs * intSize)]) :
          emitE e
        emitE (SELECT i v x e) =
          LocalSet x (Load (i * intSize) (emitV v)) :
          emitE e
        emitE (APP v vs) =
          [ Return_Call_Indirect
            (Type (typePrefix ++ show (length vs)))
            (emitV v) (map emitV vs)]
        emitE (ADD v1 v2 x e) =
          LocalSet x (Primop PLUS [emitV v1, emitV v2]) :
          emitE e
        emitE (DONE v) = [emitV v]

        emitV (NUM i) = I32_Const i
        emitV (VAR x) = LocalGet x
        emitV (LABEL f) = I32_Const $ fromJust $ f `elemIndex` fns
\end{lstlisting}
