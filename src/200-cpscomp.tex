% !TEX root = document.tex

\chapter{\label{chap:cpscomp}Compiling with Continuations}
In this chapter we will develop the LamToWat compiler that translates the lambda calculus into WebAssembly \autocite{webassemblyhomepage}. Our compiler will be written in Haskell \autocite{haskellhomepage}. We follow a minimal version of the approach by \citeauthor{DBLP:books/daglib/0022396} \autocite{DBLP:books/daglib/0022396}. Consequently, \ac{CPS} will serve as \ac{IR}. The purpose of building LamToWat is to examine and then improve its \ac{IR}. The complexity of \ac{CPS} conversion and difficulties with the untyped \ac{IR} transformations will become apparent. Although we are not concerned with the peripherals of the compiler, an \ac{IR} is not used in a vacuum. We want to examine the compiler steps that map to and from the \ac{IR} as well as the ones that map to the \ac{IR} itself. If we can improve upon the already favorable features of \ac{CPS}, we are creating a better IR to program in.

What makes \ac{CPS} favorable as an \ac{IR} is that it makes control flow and data flow explicit. These features nicely represent the objective of a compiler: translating from a high-level to a low-level language by describing abstractions in finer detail. \ac{CPS} uses special functions, called 'continuations', to describe more complex control flow constructs. The fact that continuations are special functions becomes essential when translating languages with first-class functions.

LamToWat is a multipass compiler. This means that multiple passes over the \ac{IR} will be made, each transforming a part of the representation. The complete compiler comprises these \ac{IR} transformations with additional passes to read in source files and print the resulting \ac{WA}.

\begin{gather*}
  String \xto{lex} Token \xto{parse} Exp \xto{cpsify}\\
  Cexp \xto{closify} Cexp \xto{hoist} Cexp \xto{emit} Wmodule \xto{show} String
\end{gather*}

LamToWat is split into a front-end and a back-end. The front-end of the compiler is made up of lexing, parsing, and \ac{CPS} conversion. We will not discuss lexing and parsing in detail, as it is irrelevant to the research in this thesis. We use Alex \autocite{haskellalex} and Happy \autocite{haskellhappy} to generate both a lexer and a parser for our compiler, respectively. The lexer converts source files into tokens and the parser in turn converts these tokens into an \ac{AST}. The back-end of LamToWat consist of three transformations on the \ac{IR}: closure conversion, hoisting, and emitting.

LamToWat is validated by compiling programs written in the lambda calculus to \ac{WA} and then running this code using \ac{WABT}. We compare the results of the programs to expected results and see if they match. To automate this processs we use Shake \autocite{shake}. We will use the same programs to test both versions of LamToWat.

The following sections will first describe the data types used in LamToWat with extra attention to the \ac{CPS} \ac{IR}. Then we will adhere to the organization of the compiler and discuss the relevant compiler passes over these data types in order.

\section{\label{section:datatypes}Data Types}
The data types of LamToWat will be implemented as Haskell data types. Haskell data types are indicated with the \icode{data} keyword. Every data type has a name, which comes after the keyword. Then a number of constructors follow separated by vertical bars. Data types can be recursively defined. Type aliases indicated with the \icode{type} keyword are also used; providing new names for existing data types.

\subsection{\label{subsection:expdata}Lambda Calculus}
The \icode{Exp} data type represents the \ac{AST} of the lambda calculus. The three constructors for \icode{Abs}traction, \icode{App}lication, and \icode{Var}iables encompass the standard definition of the lambda calculus. Two constructors for \icode{Num}bers and \icode{Add}ition are added to make basic arithmetic possible. the lambda calculus is one of the smallest functional languages that still has many of the interesting properties of functional languages, such as first-class functions and recursion. Moreover, the lambda calculus is well-studied \autocite{barendregt1984lambda}. Programs written in the lambda calculus will be used for testing LamToWat.

\begin{lstlisting}[language=Haskell]
type Var = String

data Exp
  = Abs Var Exp
  | App Exp Exp
  | Var Var
  | Num Int
  | Add Exp Exp
\end{lstlisting}

Two programs will be used as running examples throughout this thesis to show how the compilation process works. In the lambda calculus they are written as follows.

\begin{lstlisting}[language=Haskell]
(\ x -> x + 1) 41
(\ x -> (\ y -> x + y)) 13 29
\end{lstlisting}

We use the same notation as Haskell for the lambda calculus, \icode{\} and \icode{->} are used for lambda abstraction and a space is used for application. Both programs are valid inputs to LamToWat and contain all five of our the lambda calculus constructs. The first program applies a function that adds 1 to its argument \icode{x} to the number 42. The second programs applies a nested function that adds its arguments to the numbers 13 and 29. We choose a program with nested functions and a program without nested functions as examples to display the compilation of this language feature separately. Translated to \icode{Exp} the programs are written as follows.

\begin{lstlisting}[language=Haskell]
App (Abs "x" (Add (Var "x") (Num 1))) (Num 41)
App (App (Abs "x" (Abs "y" (Add (Var "x") (Var "y")))) (Num 13)) (Num 29)
\end{lstlisting}

You can instantly see why we prefer to write in the Haskell notation. \icode{Exp}ressions contain lots of brackets and constructor names. These declare the order of operations and the sort of (sub)-expression we are dealing with. This information is required by the compiler.

\subsection{\label{subsection:cpsdata}CPS Language}
We follow the \ac{CPS} language data type definition by \citeauthor{DBLP:books/daglib/0022396} with some minor adjustments. It is used as \ac{IR} because it facilitates compiler transformations relevant to functional programming languages. The data type is similar to the control flow graph of a program. Control flow is modeled with functions and function calls. Data flow is modeled with records. LamToWat uses \ac{CPS} to transform nested, higher-order functions to simple functions.

\icode{Cexp} functions are made up of a name, argument names, and a body. They are defined in mutually recursive \icode{FIX}ed function blocks. Values are either \icode{NUM}bers, \icode{VAR}iables, or \icode{LABEL}s. \icode{LABEL}s are used for proper function names appearring in \icode{FIX}es, \icode{VAR}iables for everything else including continuations. The \ac{CPS} language enforces the distinction between \icode{Val}ues and \icode{Cexp}ressions by putting them in separate data types. All expressions except \icode{APP} and \icode{DONE} have a \icode{Cexp} continuation as their last argument. Since the \icode{ADD}, \icode{RECORD}, and \icode{SELECT} expressions produce a value, they name it before continuing. \icode{DONE} helps us to define the function that translates a source language into \ac{CPS}. As this expression represents a return statement, it violates the second property of \ac{CPS}. No problems are cause by it, because we will make sure that it is the final expression of our programs. This makes it a global return of the program itself, not of a function.

Variable names in \icode{Cexp} are represented by simple strings. This is a design choice as it allows for nonsense programs that use undefined variables, but also for easy manipulation of variables and generation of variable names.

\begin{lstlisting}[language=Haskell]
type Var = String

data Val
  = VAR Var
  | NUM Int
  | LABEL Var

type Fun = (Var, [Var], Cexp)
data Cexp
  = FIX [Fun] Cexp
  | APP Val [Val]
  | ADD Val Val Var Cexp
  | RECORD [Val] Var Cexp
  | SELECT Int Val Var Cexp
  | DONE Val
\end{lstlisting}

The \icode{Cexp} data type guarantees that compiler transformations can be easily implemented by enforcing the following properties. \ac{CPS} also requires all functions to have an extra continuation argument. This requirement is not enforced by the data type, but needs to be fulfilled in order for it to be proper \ac{CPS}.

\begin{itemize}
\item Functions do not return, instead the last thing a function does is call a continuation.
\item Function parameters can only be values.
\item All intermediate values have names.
\end{itemize}

To illustrate how the data type is used we will translate the lambda calculus example \icode{(\ x -> x + 1) 41} to a \icode{Cexp}. We define a function in a \icode{FIX} block with name \icode{f} and argument \icode{x}. The body of this function adds \icode{x} and 1, names the result \icode{r}, and returns this result. Finally, we apply \icode{f} to 42.

\begin{lstlisting}[language=Haskell]
  FIX [("f", ["x"], ADD (VAR "x") (NUM 1) "r" (DONE "r"))] (APP (VAR "f") [NUM 41])
\end{lstlisting}

\icode{Cexp}ressions can become large and unwieldy, especially in later stages of LamToWat. For this reason we use a simpler Python-like notation to describe \icode{Cexp} from now on. LamToWat prints \icode{Cexp} with the same notation. In this notation the translated example is written as follows.

\begin{lstlisting}[language=Haskell]
def f(x):
  r <- x + 1
  r
f(41)
\end{lstlisting}

\icode{FIX} blocks have been replaced by indented \icode{def}s. The assignment of intermediate variables is indicated with a \icode{<-}. Function application is performed by placing braces around the arguments and commas in between arguments. Records will be created with square brackets \icode{[1,2,3]} and selecting from records will be done with the operator \icode{!!}. Variables generated by the compiler will be prefixed with a '\icode{.}'. Variables indicating function names will be prefixed with a '\icode{_}'.

\subsection{\label{subsection:webdata}WebAssembly Language}
The WebAssembly language is a typed, hierarchical, high-level assembly language and a subset of WebAssembly \autocite{webassemblyhomepage}. As the final language in the compilation process it gives some guarantees about the well-formedness of the output of the compiler. The fundamental unit of code is a \icode{Wmodule} made up of a list of \icode{Wtexp}s. A \icode{Wtexp} is made up of \icode{Wexp}s.

We have chosen \ac{WA} as the output language of LamToWat, because it is a 'real' language without first-class functions. A 'real' language is a language that has a sizeable group of users and developers. \ac{WA} is supported by many browsers, a large body of documentation is readily available \autocite{webassemblydocs}, and a development toolkit can be used for free \autocite{wabt}. The absence of first-class functions requires LamToWat to actually compile the lambda calculus, which does have this feature. 

\icode{I32_Const}, 32-bit integers are the basic values, used as both numbers and pointers. \icode{Func}tions are defined at the module level and can not be nested. They take zero or more arguments to a result as declared in their type signature. Local variables used in a function's body are also declared in its signature and can be \icode{LocalSet} and \icode{LocalGet}. All function calls are indirect, meaning that the function a function index points to is resolved at runtime. The type of the called function needs to be declared as the first argument to an indirect call. A \icode{Table} with function \icode{Elem}ents maps function indices to actual functions. Persistent \icode{Memory} is allocated statically and its data can be \icode{Load}ed and \icode{Store}d. A \icode{GlobalDef}ined variable can be used in the entire module, it has a \icode{GlobalType} and can be \icode{GlobalSet} and \icode{GlobalGet} if its type is \icode{Mut}able. The only \icode{Primop}eration is addition. \icode{Export} provides a function with an alias as entry point to the module.

\begin{lstlisting}[language=Haskell]
type Size = Int
type Offset = Int
type Var = String

data Wtype = I32
data GlobalType = Mut Wtype | Immut Wtype

data Wmodule = Wmodule [Wtexp]

data Wtexp
  = Func Var [(Var, Wtype)] Wtype [(Var, Wtype)] [Wexp]
  | TypeDef Var [Wtype] Wtype
  | GlobalDef Var GlobalType Wexp
  | Memory Size
  | Table Size
  | Elem Wexp [Var]
  | Export Var Var

data Wexp
  = I32_Const Int
  | Return Wexp
  | LocalGet Var
  | LocalSet Var Wexp
  | GlobalGet Var
  | GlobalSet Var Wexp
  | Type Var
  | Call_Indirect Wexp Wexp [Wexp]
  | Return_Call_Indirect Wexp Wexp [Wexp]
  | Primop Wop [Wexp]
  | Load Offset Wexp
  | Store Offset Wexp Wexp

data Wop = PLUS
\end{lstlisting}

Our example program \icode{(\ x -> x + 1) 41} would be translated to a \icode{Wmodule} as follows. We declare a \icode{Table} with one function pointer \icode{Elem}ent for function \icode{f}. We need one type for function \icode{f} called \icode{t}. It has one 32-bit integer argument and returns a 32-bit integer. We define the function \icode{f} with argument \icode{x}. The body of the function uses the \icode{Primop} \icode{PLUS} to add its argument and the number 1. The function \icode{start} is the entry point of the program. We see how an indirect call works: first state the \icode{Type}, then the index of the function in the \icode{Table}, and finally the list of arguments.

\begin{lstlisting}[language=Haskell]
Wmodule [
  Table 1,
  Elem (I32_Const 0) ["f"],
  TypeDef "t" [I32] I32,
  Func "f" [("x", I32)] I32 [] [
    Primop PLUS [LocalGet "x", I32_Const 1]
  ],
  Func "start" [] I32 [] [
    Call_Indirect (Type "t") (I32_Const 0) [I32_Const 41]
  ],
  Export "start" "start"
]
\end{lstlisting}

We will also use the printed version of \ac{WA}, which is the actual final output of our compiler. In this notation the example would be written as follows. Most of the printed notation maps one-to-one to the original. Keywords are now lower case, and square brackets are replaced with round brackets. Some keywords are added, such as \icode{funcref}, \icode{result}, and \icode{param}. Variables are prefixed with a \icode{$}. 

\begin{lstlisting}[language=Haskell]
(module
  (table 1 funcref)
  (elem (i32.const 0) $f)
  (type $t (func (param i32) (result i32)))
  (func $f (param x i32) (result i32)
    (i32.add (local.get $x) (i32.const 1)))
  (func $start (result i32)
    (call_indirect (type $t) (i32.const 0) (i32.const 41)))
  (export "start" (func "start")))
\end{lstlisting}

The creation of records is now done by storing elements on the heap and then incrementing the global \icode{$.heap_pointer}. Creating a record \icode{.record <- [1,2,3]} is handled as follows. Note that we take offsets of 4 bytes between elements, the size of a 32-bit integer. The \icode{$.heap_pointer} will keep increasing during the execution of the program as there is no garbage collection. This may lead to the \icode{$.heap_pointer} pointing to a memory location outside the allocated memory and the program crashing.

\begin{lstlisting}[language=Haskell]
(i32.store offset=0 (global.get $.heap_pointer) (i32.const 1))
(i32.store offset=4 (global.get $.heap_pointer) (i32.const 2))
(i32.store offset=8 (global.get $.heap_pointer) (i32.const 3))
(local.set $.record (global.get $.heap_pointer))
(global.set $.heap_pointer (i32.add (global.get $.heap_pointer) (i32.const 12)))
\end{lstlisting}

\section{\label{section:transforms}Transformations}
Transformations or compiler passes of LamToWat will be implemented as Haskell functions. These functions will transform the data types of the previous section. A transformation can convert a data type to a different data type, or to the same data type with certain properties.

\subsection{\label{subsection:cpsconvert}CPS Conversion}
\ac{CPS} conversion will transform the \icode{Exp} data type into the \icode{Cexp} data type. By converting to \ac{CPS} we make control flow and data flow explicit. In practice this means that we generate fresh variable names for intermediate values and have an extra argument to our conversion function that indicates the order of expressions. Fresh variables are generated using a monad \autocite{DBLP:conf/lics/Moggi89}. We use Haskell's \icode{do}-notation to write our monadic conversion function in an imperative style. The extra argument to our function is also called a metacontinuation: a continuation in the metalangauge.

\icode{convert} is split up into cases for each constructor of \icode{Exp}. The values \icode{Var} and \icode{Num} are converted to \ac{CPS} values. To convert the anonymous 
function \icode{Abs} we generate fresh variables \icode{f} and \icode{k} to name the function and its continuation. We pass the function name to the metacontinuation \icode{c}. Then we convert the body of the function \icode{e}, giving it a metacontinuation which calls the continuation with the result \icode{z}. Finally, we construct a \icode{FIX} to contain our converted function and update continuation \icode{c'}. The \icode{App} and \icode{Add} cases follow a similar approach.

\begin{lstlisting}[language=Haskell]
convert :: Exp -> (Val -> Gen Cexp) -> Gen Cexp
convert (Var x) c = c (VAR x)
convert (Num i) c = c (NUM i)
convert (Abs x e) c = do
  f <- fresh "f"
  k <- fresh "k"
  c' <- c (LABEL f)
  cf <- convert e (\ z ->
         return $ APP (VAR k) [z])
  return $ FIX [(f,[x,k],cf)] c'
convert (App f e) c = do
  r <- fresh "r"
  x <- fresh "x"
  c' <- c (VAR x)
  cf <- convert f (\ f' ->
         convert e (\ e' ->
           return $ APP f' [e', LABEL r]))
  return $ FIX [(r,[x],c')] cf
convert (Add a b) c = do
  x <- fresh "x"
  c' <- c (VAR x)
  convert a (\ a' ->
   convert b (\ b' ->
    return $ ADD a' b' x c'))
\end{lstlisting}

We give an example of a call to the \ac{CPS} conversion function to see how it works. It takes a simple \icode{Exp} and prints out a \icode{Cexp}. Variables generated by the \icode{convert} are prefixed with a '\icode{.}'. Variables indicating function names are prefixed with an additional '\icode{_}'. The function \icode{.r0} is a return point created by \icode{convert} in the \icode{App} case for the function \icode{.f2}. It is simply the identity which has no continuation argument, and is the last function that is called by the program. The anonymous function is named \icode{.f2}. It still has its argument \icode{x}, but also an extra continuation argument \icode{.k3}. We add 1 to the argument \icode{x} and assign it to \icode{.x4}, which is passed to the continuation \icode{.k3}. A function call to \icode{.f2} with arguments 41 and continuation \icode{_.r0} is the final expression of the program.

\begin{lstlisting}[language=Haskell]
> cpsify (App (Abs "x" (Add (Var "x") (Num 1))) (Num 41))

def .r0(.x1):
  .x1
def .f2(x,.k3):
  .x4 <- x + 1
  .k3(.x4)
_.f2(41,_.r0)
\end{lstlisting}

We will also take a look at an example with nested functions. We see that two return point functions are created: \icode{.r0} and \icode{.r2}. The two nested functions of the original \icode{Exp} are named \icode{.f4} and \icode{.f6}. Control flows from \icode{.f4} to \icode{.r2} to \icode{.f6} to \icode{.r0}. The variable \icode{x} is in scope when the nested function \icode{.f6} is called. In the next transformation we will package \icode{x} with the function \icode{.f6} to create a closure.

\begin{lstlisting}[language=Haskell]
> cpsify (App (App (Abs "x" (Abs "y" (Add (Var "x") (Var "y")))) (Num 13)) (Num 29))

def .r0(.x1):
  .x1
def .r2(.x3):
  .x3(29,_.r0)
def .f4(x,.k5):
  def .f6(y,.k7):
    .x8 <- x + y
    .k7(.x8)
  .k5(_.f6)
_.f4(13,_.r2)
\end{lstlisting}

\subsection{\label{section:closconvert}Closure Conversion}
Closure conversion transforms the \icode{Cexp} data type into the \icode{Cexp} data type where functions do not contain free variables. The free variables of a function are passed to the function via an extra argument. We will package a function with its free variables, this package is called a 'closure'. \icode{Cexp} implements closures as records, where the first argument is a function pointer (implemented as a \icode{LABEL}) and the following elements are free variables. The helper function \icode{lookdvs} maps function names to their free variables.

\icode{closify} transforms function \icode{APP}lication by creating closures for the called function and all its arguments. If the called function is a closure, we extract the function name and then apply it to both the closure itself and its original arguments. Otherwise, we simply add an environment argument. We rename all function arguments to closure arguments by postfixing them with the \icode{closTag}. \icode{FIX}ed function definitions are prefixed with an extra \icode{envName} argument; the body of a function begins by \icode{open}ing the environment argument, thus putting all free variables in scope. The \icode{RECORD}, \icode{SELECT}, and \icode{ADD} expressions remain the same, only their continuation is transformed.

\begin{lstlisting}[language=Haskell]
closify :: Cexp -> Cexp
closify e = go e
  where go (RECORD vs x e)  = RECORD vs x (go e)
        go (SELECT i v x e) = SELECT i v x (go e)
        go (ADD v1 v2 x e)  = ADD v1 v2 x (go e)
        go (DONE v)         = close [v] $ DONE (renameC v)
        go (APP v vs)       = close (v:vs) $ (apply v) (map renameC vs)
        go (FIX fs e)       = FIX (map goFunc fs) (go e)
        goFunc (f,as,b)     = (f,envName:as,open (lookdvs f) (go b))

        close []           k = k
        close (LABEL f:vs) k =
          RECORD (LABEL f : map VAR (lookdvs f)) (f ++ closTag) (close vs k)
        close (_:vs)       k = close vs k

        open xs k = foldr (\ (i,x) -> SELECT i (VAR envName) x) k (zip [1..] xs)

        apply (LABEL f) vs = APP (LABEL f) (VAR (f ++ closTag):vs)
        apply (VAR f)   vs = SELECT 0 (VAR f) f_F (APP (VAR f_F) (VAR f:vs))
          where f_F = f ++ funcTag
\end{lstlisting}

We show how the example function \icode{(\ x -> (\ y -> x + y)) 13 29} is closure converted. We see that we first call \icode{cpsify} and then call \icode{closify}. The output of the previous cps conversion step is transformed. It is insightful to look at the output of the previous step and compare it to the one presented here. Closures are created for all functions and sufffixed with \icode{_C}. The opening of closures to extract a function pointer is done by selecting the first argument of the closure and suffixing the name with \icode{_F}. The variable \icode{x} is now brought into the scope of the function \icode{.f6} by opening the closure argument. The function \icode{.f6} thus no longer has free variables. This was the only requirement for closure conversion to be considered succesful, but since the conversion is not optimal multiple other expressions are added too. For example, closures are created for functions that do not have any free variables, such as in the expression \icode{.f4_C <- [_.f4]}.

\begin{lstlisting}[language=Haskell]
> closify (cpsify
    (App (App (Abs "x" (Abs "y" (Add (Var "x") (Var "y")))) (Num 13)) (Num 29)))

def .r0(.env,.x1):
  .x1
def .r2(.env,.x3):
  .r0_C <- [_.r0]
  .x3_F <- .x3 !! 0
  .x3_F(.x3,29,.r0_C)
def .f4(.env,x,.k5):
  def .f6(.env,y,.k7):
    x <- .env !! 1
    .k5 <- .env !! 2
    .x8 <- x + y
    .k7_F <- .k7 !! 0
    .k7_F(.k7,.x8)
  .f6_C <- [_.f6,x,.k5]
  .k5_F <- .k5 !! 0
  .k5_F(.k5,.f6_C)
.f4_C <- [_.f4]
.r2_C <- [_.r2]
_.f4(.f4_C,13,.r2_C)
\end{lstlisting}

\subsection{\label{subsection:hoist}Hoisting}
Hoisting transforms the \icode{Cexp} data type into the \icode{Cexp} data type consisting of a single \icode{FIX}ed function definition. Hoisting is a non-local transformation: the entire expression tree is affected. \icode{hoist} makes heavy use of recursion. The only interesting case is the one for \icode{FIX}. Here we hoist and concatenate the bodies of all functions. The final continuation \icode{e} is hoisted and appended to the new list of functions.

\begin{lstlisting}[language=Haskell]
hoist :: Cexp -> Cexp
hoist (RECORD rs x e)  = case hoist e of FIX fs e -> FIX fs (RECORD rs x e)
hoist (SELECT i v x e) = case hoist e of FIX fs e -> FIX fs (SELECT i v x e)
hoist (ADD v1 v2 x e)  = case hoist e of FIX fs e -> FIX fs (ADD v1 v2 x e)
hoist (APP v vs)       = FIX [] (APP v vs)
hoist (DONE v)         = FIX [] (DONE v)               
hoist (FIX fs e)       = case hoist e of FIX fs e -> FIX (fs'++fs) e
  where fs' = concatMap (\ (f,as,b) -> case hoist b of FIX fs e -> (f,as,e) : fs) fs
\end{lstlisting}

We continue with the example from the previous subsection. The only change is that the function \icode{.f6} has been hoisted to the top level.

\begin{lstlisting}[language=Haskell]
> hoist (closify (cpsify
    (App (App (Abs "x" (Abs "y" (Add (Var "x") (Var "y")))) (Num 13)) (Num 29))))

def .r0(.env,.x1):
  .x1
def .r2(.env,.x3):
  .r0_C <- [_.r0]
  .x3_F <- .x3 !! 0
  .x3_F(.x3,29,.r0_C)
def .f4(.env,x,.k5):
  .f6_C <- [_.f6,x,.k5]
  .k5_F <- .k5 !! 0
  .k5_F(.k5,.f6_C)
def .f6(.env,y,.k7):
  x <- .env !! 1
  .k5 <- .env !! 2
  .x8 <- x + y
  .k7_F <- .k7 !! 0
  .k7_F(.k7,.x8)
.f4_C <- [_.f4]
.r2_C <- [_.r2]
_.f4(.f4_C,13,.r2_C)
\end{lstlisting}

\subsection{\label{section:emit}Emitting}
Emitting transforms the \icode{Cexp} data type into the \icode{Wmodule} data type. Here we only show the transformation of the function body \icode{emitE} and the transformation of values \icode{emitV} for brevity. Note that the case for \icode{FIX} is missing, because our entire \icode{Cexp} is a single \icode{FIX}. The case for \icode{RECORD} shows how persistent memory is implemented using a global \icode{heapPointer}. When we encounter a \icode{RECORD} we assign its elements \icode{vs} to the location of the \icode{heapPointer} with an evenly spaced offset of \icode{intSize}. Then we assign the \icode{heapPointer} to our record name and increase the \icode{heapPointer} to point the location after the record. The case for \icode{APP}lication shows that we need too use a \icode{Type} when calling a function indirectly. \icode{LABEL}s are the only \icode{Val}ue that has an interesting tranformation. It is translated to a 32-bit integer index by looking up that index in a list of functions \icode{fns}.

\begin{lstlisting}[language=Haskell]
emitE :: Cexp -> [Wexp]
emitE (RECORD vs x e) =
  zipWith (\ v i -> Store i (GlobalGet heapPointer) (emitV v)) vs [0,intSize..] ++
  LocalSet x (GlobalGet heapPointer) :
  GlobalSet heapPointer (Primop PLUS [GlobalGet heapPointer, I32_Const (length vs * intSize)]) :
  emitE e
emitE (SELECT i v x e) =
  LocalSet x (Load (i * intSize) (emitV v)) :
  emitE e
emitE (APP v vs) =
  [ Return_Call_Indirect
    (Type (typePrefix ++ show (length vs)))
    (emitV v) (map emitV vs)]
emitE (ADD v1 v2 x e) =
  LocalSet x (Primop PLUS [emitV v1, emitV v2]) :
  emitE e
emitE (DONE v) = [emitV v]

emitV :: Val -> Wexp
emitV (NUM i) = I32_Const i
emitV (VAR x) = LocalGet x
emitV (LABEL f) = I32_Const $ fromJust $ f `elemIndex` fns
\end{lstlisting}

The example shows the final output of LamToWat of the function \icode{(\ x -> x + 1) 41}. The \ac{WA} \icode{module} first declares global \icode{memory} and a global \icode{$.heap_pointer}. A \icode{table} with function pointers is made. It has as elements the functions \icode{$.r0} and \icode{$.f2}. The \icode{$.start} function is the entry point of our program and made accessible by an \icode{export} statement. The actual function we have translated is the last \icode{func}tion \icode{$.f2}. It has an extra closure argument \icode{$.env} and an extra continuation argument \icode{$.k3}. \icode{i32} is the return type, this is the return type of all functions. The local variables \icode{$.x4} and \icode{.k3_F} are also declared in the function signature.

\begin{lstlisting}[language=Haskell]
> emit (hoist (closify (cpsify (App (Abs "x" (Add (Var "x") (Num 1))) (Num 41)))))

(module  
  (memory 1)
  (global $.heap_pointer (mut i32) (i32.const 0))
  (table 2 funcref)
  (elem (i32.const 0) $.r0 $.f2)
  (type $.t2 (func (param i32) (param i32) (result i32)))
  (type $.t3 (func (param i32) (param i32) (param i32) (result i32)))
  (func $.start (result i32) (local $.f2_C i32) (local $.r0_C i32)  
    (i32.store offset=0 (global.get $.heap_pointer) (i32.const 1))
    (local.set $.f2_C (global.get $.heap_pointer))
    (global.set $.heap_pointer (i32.add (global.get $.heap_pointer) (i32.const 4)))
    (i32.store offset=0 (global.get $.heap_pointer) (i32.const 0))
    (local.set $.r0_C (global.get $.heap_pointer))
    (global.set $.heap_pointer (i32.add (global.get $.heap_pointer) (i32.const 4)))
    (return_call_indirect
      (type $.t3) (local.get $.f2_C) (i32.const 41) (local.get $.r0_C) (i32.const 1))
  )
  (export ".start" (func $.start))
  (func $.r0 (param $.env i32) (param $.x1 i32) (result i32)  
    (local.get $.x1)
  )
  (func $.f2 (param $.env i32) (param $x i32) (param $.k3 i32) (result i32)
                                                (local $.x4 i32) (local $.k3_F i32)  
    (local.set $.x4 (i32.add (local.get $x) (i32.const 1)))
    (local.set $.k3_F (i32.load offset=0 (local.get $.k3)))
    (return_call_indirect (type $.t2) (local.get $.k3) (local.get $.x4) (local.get $.k3_F))
  )
)
\end{lstlisting}
