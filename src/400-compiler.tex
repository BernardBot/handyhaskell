\chapter{\label{chap:compiler}Compiler Validation and Extension}
In this section we will discuss how we validated that our compiler worked and the performance of the transformations and command trees. We will also address some of the compiler transformations we did not implement because they were out of scope of this thesis. The online repository with the source code for LamToWat can be found on GitHub: \url{https://github.com/BernardBot/LamToWat}.

\section{Testing}
In order to test if all transformations were performed correctly we have written interpreters for the \icode{Lam}, \icode{Cps}, \icode{Wat}, and \icode{Tps} languages/data types. This allowed us to test programs at different steps of the compilation process. We compare the interpreter's results and check that they are the same. Of course this does not guarantee equality between programs. We could obtain the expected result by a wrong calculation, or by using other effects.

To automate the testing process we use Cabal \autocite{haskellcabal}, which allows us to create a test suite. By simply executing the \icode{cabal test} command we run all automated tests. A folder of lambda calculus source files that are used for testsing is provided in the projects repository.

\section{Performance}
How do command trees perform in comparison with monad transformers? The paper \citetitle{DBLP:conf/haskell/KiselyovI15} \autocite{DBLP:conf/haskell/KiselyovI15} compares the \icode{mtl} library with their own extensible effects library based on the freer monad. Their results show that algebraic effects outperform monad transformers when we nest multiple effects. Since the Haskell compiler GHC has specific optimizations for the \icode{mtl} library and especially the \icode{State} monad, monad transformers are sometimes faster for single effects.

In order to compile command trees we have used separate handlers (\icode{hClos,hRecord,hFix}). This is quite inefficient, because we have to build intermediate trees and traverse the entire data structure for each handler. Instead, we can fuse \autocite{DBLP:conf/mpc/WuS15} a sequence of handlers and remove both these performance pain points. Fusion is not yet implemented for the current command trees, but may lead to a significant performance gain.

While testing both versions of LamToWat no noticeable difference was observed for both compilation and execution time of the generated programs. Since programs were comparatively small, this does not give a good indication of how command trees would perform on larger bodies of code.

\section{\label{section:othert}Other Transformations}
Our compiler is a very simplified version of the compiler described in \citetitle{DBLP:books/daglib/0022396} \autocite{DBLP:books/daglib/0022396}. \citeauthor{DBLP:books/daglib/0022396} discusses a number of other transformations that improve the performance of the generated code and compile other language features:

\begin{itemize}
\item Closure Optimization\\\\
Creating efficient closures is not trivial. LamToWat makes simple but inefficient closures. In order to create closures that are optimized for speed or memory usage we will need to perform variable analysis. We could create a command that is a special form of \icode{Fix} that carries this variable information with it. We can then implement two transformations: one that does the analysis and one that creates the better closures. The syntactic nature of \icode{Tps} allows us to do this optimization.
\item Compilation of pattern matching\\\\
Pattern matching is a feature that benefits many functional languages. To compile pattern matching we can use \icode{switch} expressions. \icode{switch} expressions takes a value and a list of expressions. We would need to add a switch expression to \icode{Cps} in the first version of LamToWat and a switch command to both \icode{Tree} and \icode{Tps}. Our source language would have \icode{case} statements. These would need to be compiled to \icode{switch} expressions. This translation can be optimized by way of a decision-tree algorithm.
\item Inlining funcions\\\\
The inlining of functions substitutes a function body for a function call. This increases performance of a program. Substitution of function bodies may be problematic if not performed correctly, because it may lead to a large increase in program size. To implement this feature in LamToWat we could create a new command that identifies function bodies as candidates for inline expansion. We would have one command for candidate functions and one for non-candidate functions. We may need an auxilary function to perform the candidate analysis.
\end{itemize}
