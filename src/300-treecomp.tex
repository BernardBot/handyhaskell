% !TEX root = document.tex

\chapter{\label{chap:treecomp}Compiling with Command Trees}
The version of the LamToWat compiler described in the previous section has two major flaws that hinder language implementers and compiler writers:

\begin{itemize}
\item \ac{CPS} conversion is hard to write,
\item \ac{IR} transformations are untyped.
\end{itemize}

In this chapter we propose two data types called semantic and syntactic command trees to solve these two shortcomings, respectively. They will replace the original \ac{CPS} \ac{IR}.

We elaborate on the two compiler flaws. A programmer needs to have a decent understanding of \ac{CPS} to implement the \ac{CPS} conversion. This is because the \ac{CPS} conversion function itself is written in \ac{CPS}. LamToWat uses a (meta)continuation to specify control flow. We would like a programmer that just wants to implement a language to not have to worry about the inner workings of the \ac{IR}. We need an extra layer of abstraction that makes it easy to write: first convert this expression, then convert that expression, then return a final expression. We want the flexibility of continuations without having to deal with their complexity directly.

The transformations in the back-end of the LamToWat compiler change the structure and properties of the \ac{IR}. The \icode{Cexp} data type gives no indication of these changes. A programmer that is implementing a compiler would like some assurance that the transformations she has specified guarantee those properties. Additionally, only a part of the \ac{IR} is transformed, the other constructors remain the same and merely need to be traversed, i.e., their continuation argument needs to be transformed. \icode{Cexp} needs explicit traversal of these constructors. We want to group parts of the \ac{IR} together as to make the transformation contain less boilerplate.

\section{\label{section:commandtree}Command Trees}
Command trees are a data type used to sequence commands. A well known data type that is also capable of sequencing is a list. Command trees improve upon lists by adding types that restrict its structure, and by providing the ability to use the result of one command in the commands after that. Just like lists, command trees are monads. This allows us to bind command trees together easily and write functions using \icode{do}-notation. Commands themselves are a syntactic representations of a monadic computations. In simpler terms this can be described as the meaning of a command is something that is left to the programmer. What a command does is implemented later in a function sometimes called a handler.

Command trees are implemented using Haskell \acp{GADT}. A \ac{GADT} gives us the power to add more types to our \ac{IR}. A \ac{GADT} has a somewhat different notation than a normal data type. After the \icode{data} keyword we specify a name and type parameters, a \icode{where} keyword follows. Then the different constructors of the \ac{GADT} are declared. The constructors are given a name followed by \icode{::}. The types of the arguments follow, separated by \icode{->}. Finally, we declare the constructor with its type parameters, which can relate to the types of the arguments.

\subsection{\label{subsection:semantree}Semantic Command Trees}
\begin{lstlisting}[language=Haskell]
data Tree sig a where
  Leaf :: a -> Tree sig a
  Node :: sig n b p r q ->
          Vec n (p -> Tree sig r) ->
          Option b (q -> Tree sig a) ->
          Tree sig a
\end{lstlisting}

The structure and monadic nature of semantic command trees have already been discussed in \citetitle{commandtreespoulsen}. A modified version of the discussion will be restated here for completeness. A semantic command tree consists of two constructors: \icode{Leaf} and \icode{Node}. \icode{Leaf v} is the smallest form of semantic command tree that exists and simply returns the value \icode{v}. A \icode{Node op vs k} semantic command tree is somewhat more complex and is made up of three parts:

\begin{itemize}
\item An operation \icode{op} with signature \icode{sig},
\item A dependently typed vector of subcontinuations \icode{vs},
\item An optional continuation (also called join-point) \icode{k}.
\end{itemize}

A semantic command tree does not know what set of commands is used. However, it does encode strong type constraints on its subcontinuations and continuation. To enforce these constraints we use a signature \icode{sig}. This way a command tells the semantic command tree what comes after itself. More formally a signature has type:

\begin{lstlisting}[language=Haskell]
type Sig :: Nat -> Bool -> * -> * -> * -> *
\end{lstlisting}

A signature instance \icode{sig n b p r q} tells us the following things:

\begin{itemize}
\item The tag of the command and its enclosed parameters.
\item The number of subcontinuations \icode{n}.
\item If the command has a continuation.
\item The argument \icode{p} and return type \icode{r} of the subcontinuations.
\item The argument type \icode{q} of the continuation.
\end{itemize}

Without commands we can not do anything with our command trees. We will use a set of commands that reflect the constructors from the \ac{CPS} datatype. We also define a number of extra commands that will help us compile. This progamifying or treeifying of a command or request is called lifting. This is called so because we 'lift' the expression into the monad.

\begin{lstlisting}[language=Haskell]
data Cmd :: Sig where
  -- cps commands
  Fix    :: Vec n (Var, [Var]) -> Cmd n       'True  ()    Val   ()
  App    :: Val -> [Val] ->       Cmd 'Z      'False Void  Void  Val 
  Plus   :: Val -> Val ->         Cmd 'Z      'True  Void  Void  Val
  Record :: [Val] ->              Cmd 'Z      'True  Void  Void  Val
  Select :: Int -> Val ->         Cmd 'Z      'True  Void  Void  Val
  -- extra compilation commands
  Get    :: Var ->                Cmd 'Z      'True  Void  Void  Val
  Put    :: Var -> Val ->         Cmd 'Z      'True  Void  Void  ()
  Block  ::                       Cmd ('S 'Z) 'True  ()    Val   Val
  Fresh  :: Var ->                Cmd 'Z      'True  Void  Void  Var
\end{lstlisting}

We can see that almost all our original constructors from the \ac{CPS} data type are here. Except the \icode{DONE} constructor will be modeled by the \icode{Leaf} constructor of the semantic command tree. The \icode{Var} and \icode{Val} type are the same as in \ac{CPS}. Our four extra commands represent mutable state, command concatenation, and fresh variable name generation.

We take a look at the signatures of the commands to see how they are structured. The \icode{Block} and \icode{Fix} command are the only ones that have subcontinuations. The \icode{Void} indicate that there is no subcontinuation. \icode{Fix} has a subcontinuation for every function definition. This is required by the constructor itself as the natural number \icode{n} appears in both the function name with arguments and subcontinuations. The \icode{App} command does not have a continuation, which will lead to some trouble when trying to concatenate it with other commands. The return type of commands indicates that commands will bind to a variable \icode{Val}, or only produce a side-effect \icode{()}.

\subsection{\label{subsection:syntree}Syntactic Command Trees}
After we have performed our \ac{CPS} conversion our code lives inside a semantic command tree. The following transformations will have to be performed on the semantic command tree. The abstract nature of semantic command trees prevents us from defining these transformations without escaping from its abstraction. The fundamental problem is that continuations are defined as functions in the metalanguage instead of syntactic constructs in the \ac{IR} itself.

\begin{lstlisting}[language=Haskell]
-- old dv function
dv (ADD _ _ x k) xs = dv k (x:xs)
-- new dv function
dv (Node (Plus _ _) Nil (Some k)) xs = _
\end{lstlisting}

The first transformation after \ac{CPS} conversion is closure conversion. To define closure conversion, we first need to write a helper function \icode{dv} that makes a map of free defined variables for each function. Our semantic command tree breaks down, because variables are part of the metalanguage. We examine the case for addition. In the old version we could update our variable list with the name \icode{x} that was part of the \icode{ADD} expression itself. In the new version our variable name is hidden inside the continuation \icode{k}, which has type \icode{Val -> Tree Cmd Val}. If we want to access our continuation, we will have to generate a fresh variable name to apply it to. However, we would need the same generated variables again in the actual closure conversion. This is simply not practical. We need a syntactic embedding of variable binding in our \ac{IR} in order to perform closure conversion.

\begin{lstlisting}[language=Haskell]
data Tree sig a where
  Leaf :: a -> Tree sig a
  Node :: sig p r ->
          [(p, Tree sig r)] ->
          Maybe (Var, Tree sig a) ->
          Tree sig a
\end{lstlisting}

We propose syntactic command trees as a solution to this problem. Syntactic command trees have, as their name implies, syntactic representation of name binding in their constructors. The dependent types are dropped to make non-local transformations of the tree manageable. Dependently typed vectors will be replaced with lists and the dependently type \icode{Option} will be replaced with \icode{Maybe}. The only types that remain in the signature are the argument \icode{p} and return type \icode{r} of the subcontinuations. We require a function to translate from our command trees (now also called semantic command trees) to syntactic command trees. Moreover, a new set of commands need to be made for the syntactic trees.

\begin{lstlisting}[language=Haskell]
type Sig = * -> * -> *

data Fun :: Sig where
  Fix :: [(Var,[Var])] -> Fun () Val

data Kon :: Sig where
  App    :: Val -> [Val] ->  Kon Void Void
  Plus   :: Val -> Val ->    Kon Void Void
  Record :: [Val] ->         Kon Void Void
  Select :: Int -> Val ->    Kon Void Void
\end{lstlisting}

The set of syntactic \ac{CPS} commands does no longer require the commands related to compilation. These commands are given a syntactic representation within the tree. Therefore, our set of commands will match the original \ac{CPS} commands almost completely with the exception of the \icode{DONE} expression. The commands are separated into two datatypes. This separation is made with the compilation steps that follow in mind, which are mainly concerned with transforming functions.

\begin{lstlisting}[language=Haskell]
infixr 5 :+:
data (:+:) :: Sig -> Sig -> Sig where
  Inl :: sig1 p r -> (sig1 :+: sig2) p r
  Inr :: sig2 p r -> (sig1 :+: sig2) p r

type Cmd = Fun :+: Kon
\end{lstlisting}

We will combine \icode{Fun} and \icode{Kon} using an open union or sum data type. An open union can be viewed as a list of data types. More precisely, it is a binary tree which has datatypes as leaf nodes. \icode{:+:} is right-associative and has two constructors \icode{Inl} and \icode{Inr}, which inject a datatype with two type variables into the left or right side of the tree, respectively. Note that we can nest instances of open unions to create open unions. In order to extract a data type from an open union we simply pattern match on the two constructors. Our new \icode{Cmd} will have \icode{Fun} in its right subtree and \icode{Kon} in its left subtree.

\section{\label{section:treensforms}Treensformations}
\subsection{\label{subsection:cpsconvert}CPS Conversion}
\begin{lstlisting}[language=Haskell]
denote :: Exp -> Tree Cmd Val
denote (Var x) = return (VAR x)
denote (Num i) = return (NUM i)
denote (Abs x e) = do
  f <- fresh ".f"
  k <- fresh ".k"
  fix ((f,[x,k],do
           e <- denote e
           app (VAR k) [e])
       ::: Nil)
  return (LABEL f)
denote (App f a) = block (do
  f <- denote f
  a <- denote a
  k <- get ".nxt"
  app f [a,k])
denote (Add a b) = do
  a <- denote a
  b <- denote b
  plus a b
\end{lstlisting}
\subsection{\label{subsection:semtosyn}Tree Compilation}
\begin{lstlisting}[language=Haskell]
comp :: S.Tree S.Cmd Val -> M (Tree Cmd Val)
comp (S.Leaf x) = return (Leaf x)
comp (S.Node (S.Fresh x) Nil (Some k)) = do
  x <- fresh x
  comp (k x)
comp (S.Node (S.Get x) Nil (Some k)) = do
  x <- get x
  comp (k x)
comp (S.Node (S.Put x v) Nil (Some k)) = do
  put x v (comp (k ()))
comp (S.Node S.Block (k' ::: Nil) (Some k)) = do
  x <- fresh ".x"
  r <- fresh ".r"
  nxt <- get ".nxt"
  k <- comp (k (VAR x))
  k' <- comp (do
     S.put ".nxt" (LABEL r)
     k' <- k' ()
     S.put ".nxt" nxt
     S.app (LABEL r) [k'])
  return (Node (Inl (Fix [(r,[x])])) [((), k)] (Just ("_", k')))
comp (S.Node (S.Fix fs)     ks  (Some k)) = do
  ks <- mapM (\ k -> comp (k ())) ks
  k <- comp (k ())
  return (Node (Inl (Fix (toList fs))) (map ((),) (toList ks)) (Just ("_", k)))
comp (S.Node (S.App v vs)   Nil None)     = do
  return (Node (Inr (App v vs)) [] Nothing)
comp (S.Node (S.Plus v1 v2) Nil (Some k)) = do
  x <- fresh ".xl"
  k <- comp (k (VAR x))
  return (Node (Inr (Plus v1 v2)) [] (Just (x, k)))
comp (S.Node (S.Record vs)  Nil (Some k)) = do
  x <- fresh ".xl"
  k <- comp (k (VAR x))
  return (Node (Inr (Record vs)) [] (Just (x, k)))
comp (S.Node (S.Select i v) Nil (Some k)) = do
  x <- fresh ".xl"
  k <- comp (k (VAR x))
  return (Node (Inr (Select i v)) [] (Just (x, k)))
\end{lstlisting}
\subsection{\label{subsection:closconvert}Closure Conversion}
\begin{lstlisting}[language=Haskell]
closify e = go e
  where go (Leaf v)                    = close [v]    $ Leaf (renameC v)
        go (Node (Inr (App v vs)) _ _) = close (v:vs) $ (apply v) (map renameC vs)
        go (Node (Inl (Fix fs)) ks (Just (_, k))) = do fix fxs; go k
          where fxs = zipWith (\ (f,as) ((),b) ->
                                 (f,envName:as,open (lookdvs f) 1 (go b))) fs ks

        go (Node op@(Inr _) ks (Just (x, k))) = (Node op ks (Just (x, go k)))

        dvs = dv e []
        lookdvs f = fromJust $ lookup f dvs

        closTag = "_C"
        funcTag = "_F"
        envName = ".env"

        renameC (LABEL f) = VAR (f ++ closTag)
        renameC v = v
        
        close [] k = k
        close (LABEL f:vs) k = do
          record (LABEL f : map VAR (lookdvs f)) (f ++ closTag)
          close vs k
        close (_:vs) k = close vs k

        open [] i k = k
        open (x:xs) i k = do
          select i (VAR envName) x
          open xs (i+1) k

        apply (LABEL f) vs = app (LABEL f) (VAR (f ++ closTag):vs)
        apply (VAR f) vs = do
          let f_F = f ++ funcTag
          select 0 (VAR f) f_F
          app (VAR f_F) (VAR f:vs)
\end{lstlisting}
\subsection{\label{subsection:hoist}Hoisting}
\begin{lstlisting}[language=Haskell]
hoist :: Tree Cmd Val -> Tree Cmd Val
hoist k@(Leaf x) = Node (Inl (Fix [])) [] (Just ("_", k))
hoist k@(Node (Inr (App _ _)) [] Nothing) =
  Node (Inl (Fix [])) [] (Just ("_", k))

hoist (Node (Inl (Fix fs)) ks (Just ("_", k))) = case hoist k of
  Node (Inl (Fix fs)) ks (Just (_, k)) ->
    Node (Inl (Fix (fs'++fs))) ((map ((),) ks')++ks) (Just ("_", k))
  where fs' :: [(Var,[Var])]
        ks' :: [Tree Cmd Val]
        (fs',ks') = unzip $ concatMap (\ ((f,as),b) -> case hoist b of
                              Node (Inl (Fix fs)) ks (Just (_, k)) ->
                                ((f,as),k) : zip fs (map snd ks))
                                (zip fs (map snd ks))

hoist (Node op@Inr{}  _ (Just (v,k))) = case hoist k of
  Node (Inl (Fix fs)) ks (Just (_, k)) ->
    Node (Inl (Fix fs)) ks (Just ("_", (Node op [] (Just (v,k)))))
\end{lstlisting}
