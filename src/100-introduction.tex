% !TEX root = document.tex

\chapter{\label{chap:introduction}Introduction}

% CPS is
% - a method of compilation
% - flexible
% - lambda calculus minus
% - an intermediate representation
\ac{CPS} is a time-tested paradigm for functional compilation \autocite{steele1978rabbit, DBLP:books/daglib/0022396}. It bridges the gap between high-level programming languages and (abstract) machine code. Although \ac{CPS} is mostly used for compiling functional languages, it is flexible enough to compile imperative languages. \ac{CPS} is a style of programming, but can also be seen as a language in itself. More specifically, a restricted form of the lambda calculus \autocite{barendregt1984lambda}. This fact gives it a solid theoretical foundation.

% IR needs to
% - represent functions
% - represent registers
% - represent data flow
% these are already satisfied by CPS, we add
% - declarativity (front-end)
% - confidence / restriction
The language that is used by a compiler to represent source code is called an \ac{IR}. This is the data structure that a compiler writer will use to implement optimizations and translations. It will need to fulfill some requirements. To compile a simple language we will require our \ac{IR} represent: functions, registers, and data flow. These requirements are already met by \ac{CPS}. In this thesis we want to improve on this baseline. Our \ac{IR} will have a declarative front-end and be intrinsically typed. The goal of these two new properties are to help two groups of programmers: language implementers and compiler writers.

% The contributions of this thesis are
% - a compiler from the lambda calculus to webassembly
% - multiple version of the IR
% - analysis...
The contributions of this thesis are the following:
\begin{itemize}
\item A WebAssembly compiler for the lambda calculus.
\item Multiple versions that show the improvements over \ac{CPS}.
\item Analysis of \ac{IR} requirements.
\end{itemize}